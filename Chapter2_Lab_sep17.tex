\documentclass[]{article}
\usepackage{lmodern}
\usepackage{amssymb,amsmath}
\usepackage{ifxetex,ifluatex}
\usepackage{fixltx2e} % provides \textsubscript
\ifnum 0\ifxetex 1\fi\ifluatex 1\fi=0 % if pdftex
  \usepackage[T1]{fontenc}
  \usepackage[utf8]{inputenc}
\else % if luatex or xelatex
  \ifxetex
    \usepackage{mathspec}
  \else
    \usepackage{fontspec}
  \fi
  \defaultfontfeatures{Ligatures=TeX,Scale=MatchLowercase}
\fi
% use upquote if available, for straight quotes in verbatim environments
\IfFileExists{upquote.sty}{\usepackage{upquote}}{}
% use microtype if available
\IfFileExists{microtype.sty}{%
\usepackage{microtype}
\UseMicrotypeSet[protrusion]{basicmath} % disable protrusion for tt fonts
}{}
\usepackage[margin=1in]{geometry}
\usepackage{hyperref}
\hypersetup{unicode=true,
            pdftitle={Chapter 2 Lab},
            pdfauthor={Allen Church},
            pdfborder={0 0 0},
            breaklinks=true}
\urlstyle{same}  % don't use monospace font for urls
\usepackage{color}
\usepackage{fancyvrb}
\newcommand{\VerbBar}{|}
\newcommand{\VERB}{\Verb[commandchars=\\\{\}]}
\DefineVerbatimEnvironment{Highlighting}{Verbatim}{commandchars=\\\{\}}
% Add ',fontsize=\small' for more characters per line
\usepackage{framed}
\definecolor{shadecolor}{RGB}{248,248,248}
\newenvironment{Shaded}{\begin{snugshade}}{\end{snugshade}}
\newcommand{\KeywordTok}[1]{\textcolor[rgb]{0.13,0.29,0.53}{\textbf{#1}}}
\newcommand{\DataTypeTok}[1]{\textcolor[rgb]{0.13,0.29,0.53}{#1}}
\newcommand{\DecValTok}[1]{\textcolor[rgb]{0.00,0.00,0.81}{#1}}
\newcommand{\BaseNTok}[1]{\textcolor[rgb]{0.00,0.00,0.81}{#1}}
\newcommand{\FloatTok}[1]{\textcolor[rgb]{0.00,0.00,0.81}{#1}}
\newcommand{\ConstantTok}[1]{\textcolor[rgb]{0.00,0.00,0.00}{#1}}
\newcommand{\CharTok}[1]{\textcolor[rgb]{0.31,0.60,0.02}{#1}}
\newcommand{\SpecialCharTok}[1]{\textcolor[rgb]{0.00,0.00,0.00}{#1}}
\newcommand{\StringTok}[1]{\textcolor[rgb]{0.31,0.60,0.02}{#1}}
\newcommand{\VerbatimStringTok}[1]{\textcolor[rgb]{0.31,0.60,0.02}{#1}}
\newcommand{\SpecialStringTok}[1]{\textcolor[rgb]{0.31,0.60,0.02}{#1}}
\newcommand{\ImportTok}[1]{#1}
\newcommand{\CommentTok}[1]{\textcolor[rgb]{0.56,0.35,0.01}{\textit{#1}}}
\newcommand{\DocumentationTok}[1]{\textcolor[rgb]{0.56,0.35,0.01}{\textbf{\textit{#1}}}}
\newcommand{\AnnotationTok}[1]{\textcolor[rgb]{0.56,0.35,0.01}{\textbf{\textit{#1}}}}
\newcommand{\CommentVarTok}[1]{\textcolor[rgb]{0.56,0.35,0.01}{\textbf{\textit{#1}}}}
\newcommand{\OtherTok}[1]{\textcolor[rgb]{0.56,0.35,0.01}{#1}}
\newcommand{\FunctionTok}[1]{\textcolor[rgb]{0.00,0.00,0.00}{#1}}
\newcommand{\VariableTok}[1]{\textcolor[rgb]{0.00,0.00,0.00}{#1}}
\newcommand{\ControlFlowTok}[1]{\textcolor[rgb]{0.13,0.29,0.53}{\textbf{#1}}}
\newcommand{\OperatorTok}[1]{\textcolor[rgb]{0.81,0.36,0.00}{\textbf{#1}}}
\newcommand{\BuiltInTok}[1]{#1}
\newcommand{\ExtensionTok}[1]{#1}
\newcommand{\PreprocessorTok}[1]{\textcolor[rgb]{0.56,0.35,0.01}{\textit{#1}}}
\newcommand{\AttributeTok}[1]{\textcolor[rgb]{0.77,0.63,0.00}{#1}}
\newcommand{\RegionMarkerTok}[1]{#1}
\newcommand{\InformationTok}[1]{\textcolor[rgb]{0.56,0.35,0.01}{\textbf{\textit{#1}}}}
\newcommand{\WarningTok}[1]{\textcolor[rgb]{0.56,0.35,0.01}{\textbf{\textit{#1}}}}
\newcommand{\AlertTok}[1]{\textcolor[rgb]{0.94,0.16,0.16}{#1}}
\newcommand{\ErrorTok}[1]{\textcolor[rgb]{0.64,0.00,0.00}{\textbf{#1}}}
\newcommand{\NormalTok}[1]{#1}
\usepackage{graphicx,grffile}
\makeatletter
\def\maxwidth{\ifdim\Gin@nat@width>\linewidth\linewidth\else\Gin@nat@width\fi}
\def\maxheight{\ifdim\Gin@nat@height>\textheight\textheight\else\Gin@nat@height\fi}
\makeatother
% Scale images if necessary, so that they will not overflow the page
% margins by default, and it is still possible to overwrite the defaults
% using explicit options in \includegraphics[width, height, ...]{}
\setkeys{Gin}{width=\maxwidth,height=\maxheight,keepaspectratio}
\IfFileExists{parskip.sty}{%
\usepackage{parskip}
}{% else
\setlength{\parindent}{0pt}
\setlength{\parskip}{6pt plus 2pt minus 1pt}
}
\setlength{\emergencystretch}{3em}  % prevent overfull lines
\providecommand{\tightlist}{%
  \setlength{\itemsep}{0pt}\setlength{\parskip}{0pt}}
\setcounter{secnumdepth}{0}
% Redefines (sub)paragraphs to behave more like sections
\ifx\paragraph\undefined\else
\let\oldparagraph\paragraph
\renewcommand{\paragraph}[1]{\oldparagraph{#1}\mbox{}}
\fi
\ifx\subparagraph\undefined\else
\let\oldsubparagraph\subparagraph
\renewcommand{\subparagraph}[1]{\oldsubparagraph{#1}\mbox{}}
\fi

%%% Use protect on footnotes to avoid problems with footnotes in titles
\let\rmarkdownfootnote\footnote%
\def\footnote{\protect\rmarkdownfootnote}

%%% Change title format to be more compact
\usepackage{titling}

% Create subtitle command for use in maketitle
\providecommand{\subtitle}[1]{
  \posttitle{
    \begin{center}\large#1\end{center}
    }
}

\setlength{\droptitle}{-2em}

  \title{Chapter 2 Lab}
    \pretitle{\vspace{\droptitle}\centering\huge}
  \posttitle{\par}
    \author{Allen Church}
    \preauthor{\centering\large\emph}
  \postauthor{\par}
    \date{}
    \predate{}\postdate{}
  

\begin{document}
\maketitle

Load necessary packages

\begin{Shaded}
\begin{Highlighting}[]
\KeywordTok{library}\NormalTok{(haven)}
\end{Highlighting}
\end{Shaded}

Set working directory and load lab data

\begin{Shaded}
\begin{Highlighting}[]
\NormalTok{lab <-}\StringTok{ }\KeywordTok{read_dta}\NormalTok{(}\StringTok{"Ch2_lab_survey_data.dta"}\NormalTok{)}
\end{Highlighting}
\end{Shaded}

\begin{enumerate}
\def\labelenumi{\arabic{enumi})}
\tightlist
\item
  Use the following to create dummy variables for Arlington and Prince
  William Counties. How many observations are from each county?
\end{enumerate}

\begin{Shaded}
\begin{Highlighting}[]
\CommentTok{#Create dummy variable for Arlington county, selecting corresponding precinct codes with OR operators}
\NormalTok{lab}\OperatorTok{$}\NormalTok{Arlington <-}\StringTok{ }\NormalTok{(}
\NormalTok{  lab}\OperatorTok{$}\NormalTok{precinct }\OperatorTok{==}\StringTok{ "AR49"} \OperatorTok{|}\StringTok{ }\NormalTok{lab}\OperatorTok{$}\NormalTok{precinct }\OperatorTok{==}\StringTok{ "AR22"} \OperatorTok{|}\StringTok{ }\NormalTok{lab}\OperatorTok{$}\NormalTok{precinct }\OperatorTok{==}\StringTok{ "AR2"} \OperatorTok{|}
\StringTok{  }\NormalTok{lab}\OperatorTok{$}\NormalTok{precinct }\OperatorTok{==}\StringTok{ "AR18"} \OperatorTok{|}\StringTok{ }\NormalTok{lab}\OperatorTok{$}\NormalTok{precinct }\OperatorTok{==}\StringTok{ "41"} \OperatorTok{|}\StringTok{ }\NormalTok{lab}\OperatorTok{$}\NormalTok{precinct }\OperatorTok{==}\StringTok{ "16"} \OperatorTok{|}
\StringTok{  }\NormalTok{lab}\OperatorTok{$}\NormalTok{precinct }\OperatorTok{==}\StringTok{ "4"} \OperatorTok{|}\StringTok{ }\NormalTok{lab}\OperatorTok{$}\NormalTok{precinct }\OperatorTok{==}\StringTok{ "17"} \OperatorTok{|}\StringTok{ }\NormalTok{(lab}\OperatorTok{$}\NormalTok{precinct }\OperatorTok{==}\StringTok{ "2"} \OperatorTok{&}
\StringTok{    }
\NormalTok{lab}\OperatorTok{$}\NormalTok{state }\OperatorTok{==}\StringTok{ }\DecValTok{4}\OperatorTok{&}\StringTok{ }\OperatorTok{!}\KeywordTok{is.na}\NormalTok{(lab}\OperatorTok{$}\NormalTok{state))}\OperatorTok{|}
\StringTok{  }\NormalTok{lab}\OperatorTok{$}\NormalTok{precinct }\OperatorTok{==}\StringTok{ "31"} \OperatorTok{|}\StringTok{ }\NormalTok{lab}\OperatorTok{$}\NormalTok{precinct }\OperatorTok{==}\StringTok{ "48"}\NormalTok{)}

\CommentTok{#Count observations from each county, in this case TRUE corresponds to Arlington}
\KeywordTok{table}\NormalTok{(lab}\OperatorTok{$}\NormalTok{Arlington)}
\end{Highlighting}
\end{Shaded}

\begin{verbatim}
## 
## FALSE  TRUE 
##  1884   475
\end{verbatim}

\begin{Shaded}
\begin{Highlighting}[]
\CommentTok{#Create dummy variable for Prince William county, selecting corresponding precinct codes with OR operators}
\NormalTok{lab}\OperatorTok{$}\NormalTok{PrinceWilliam1 <-}\StringTok{ }\NormalTok{(}
\NormalTok{  lab}\OperatorTok{$}\NormalTok{precinct }\OperatorTok{==}\StringTok{ "PW 101"} \OperatorTok{|}\StringTok{ }\NormalTok{lab}\OperatorTok{$}\NormalTok{precinct }\OperatorTok{==}\StringTok{ "PW 104"} \OperatorTok{|}\StringTok{  }\NormalTok{lab}\OperatorTok{$}\NormalTok{precinct }\OperatorTok{==}\StringTok{ "PW 401"} \OperatorTok{|}\StringTok{ }\NormalTok{lab}\OperatorTok{$}\NormalTok{precinct }\OperatorTok{==}\StringTok{ "PW101"} \OperatorTok{|}
\StringTok{  }\NormalTok{lab}\OperatorTok{$}\NormalTok{precinct }\OperatorTok{==}\StringTok{ "PW104"} \OperatorTok{|}\StringTok{ }\NormalTok{lab}\OperatorTok{$}\NormalTok{precinct }\OperatorTok{==}\StringTok{ "PW402"} \OperatorTok{|}\StringTok{ }\NormalTok{lab}\OperatorTok{$}\NormalTok{precinct }\OperatorTok{==}\StringTok{ "PW406"}  \OperatorTok{|}\StringTok{ }\NormalTok{lab}\OperatorTok{$}\NormalTok{precinct }\OperatorTok{==}\StringTok{ "401"} \OperatorTok{|}\StringTok{ }\NormalTok{lab}\OperatorTok{$}\NormalTok{precinct }\OperatorTok{==}\StringTok{ "402"} \OperatorTok{|}\StringTok{ }
\NormalTok{(lab}\OperatorTok{$}\NormalTok{precinct }\OperatorTok{==}\StringTok{ "104"} \OperatorTok{&}\StringTok{ }\NormalTok{lab}\OperatorTok{$}\NormalTok{state }\OperatorTok{==}\StringTok{ }\DecValTok{4}\NormalTok{)    )}

\CommentTok{#Count observations from each county, in this case TRUE corresponds to Prince William}
\KeywordTok{table}\NormalTok{(lab}\OperatorTok{$}\NormalTok{PrinceWilliam1)}
\end{Highlighting}
\end{Shaded}

\begin{verbatim}
## 
## FALSE  TRUE 
##  2171   188
\end{verbatim}

\begin{enumerate}
\def\labelenumi{\arabic{enumi})}
\setcounter{enumi}{1}
\tightlist
\item
  Create dummy variables for each state/DC. How many observations are in
  DC, Maryland, Ohio and Virginia?
\end{enumerate}

\begin{Shaded}
\begin{Highlighting}[]
\CommentTok{#Create dummy variables for each state}
\NormalTok{lab}\OperatorTok{$}\NormalTok{DC <-}\StringTok{ }\NormalTok{(lab}\OperatorTok{$}\NormalTok{state }\OperatorTok{==}\StringTok{ }\DecValTok{1}\NormalTok{)}
\NormalTok{lab}\OperatorTok{$}\NormalTok{Maryland <-}\StringTok{ }\NormalTok{(lab}\OperatorTok{$}\NormalTok{state }\OperatorTok{==}\StringTok{ }\DecValTok{2}\NormalTok{)}
\NormalTok{lab}\OperatorTok{$}\NormalTok{Ohio <-}\StringTok{ }\NormalTok{(lab}\OperatorTok{$}\NormalTok{state }\OperatorTok{==}\StringTok{ }\DecValTok{3}\NormalTok{)}
\NormalTok{lab}\OperatorTok{$}\NormalTok{Virginia <-}\StringTok{ }\NormalTok{(lab}\OperatorTok{$}\NormalTok{state }\OperatorTok{==}\StringTok{ }\DecValTok{4}\NormalTok{)}

\CommentTok{#Tabulate observations for each state}
\KeywordTok{table}\NormalTok{(lab}\OperatorTok{$}\NormalTok{state)}
\end{Highlighting}
\end{Shaded}

\begin{verbatim}
## 
##   1   2   3   4 
## 768 369 547 664
\end{verbatim}

\begin{enumerate}
\def\labelenumi{\arabic{enumi})}
\setcounter{enumi}{2}
\tightlist
\item
  Convert the year\_born variable into age. Be sure to check for and
  correct for lab errors. What is the average age of all observations in
  the lab set? The minimum and maximum?
\end{enumerate}

\begin{Shaded}
\begin{Highlighting}[]
\CommentTok{#Since the survey was taken in 2016, subtract 2016 - year born to obtain age}
\CommentTok{#Create new age column}
\NormalTok{lab}\OperatorTok{$}\NormalTok{age <-}\StringTok{ }\DecValTok{2016} \OperatorTok{-}\StringTok{ }\NormalTok{lab}\OperatorTok{$}\NormalTok{year_born}

\CommentTok{#The first summary of the age column shows there is a max age of 152 and 482 NA rows}
\KeywordTok{summary}\NormalTok{(lab}\OperatorTok{$}\NormalTok{age)}
\end{Highlighting}
\end{Shaded}

\begin{verbatim}
##    Min. 1st Qu.  Median    Mean 3rd Qu.    Max.    NA's 
##   17.00   30.00   41.00   43.17   55.00  152.00     482
\end{verbatim}

\begin{Shaded}
\begin{Highlighting}[]
\CommentTok{#Subset the lab dataframe and exclude values where age is above 100}
\NormalTok{newdata <-}\StringTok{ }\KeywordTok{subset}\NormalTok{(lab, age }\OperatorTok{<}\StringTok{ }\DecValTok{100}\NormalTok{)}

\CommentTok{#The summary of the newdata age column shows a new maximum of 95, which is possible}
\KeywordTok{summary}\NormalTok{(newdata}\OperatorTok{$}\NormalTok{age)}
\end{Highlighting}
\end{Shaded}

\begin{verbatim}
##    Min. 1st Qu.  Median    Mean 3rd Qu.    Max. 
##   17.00   30.00   41.00   43.07   55.00   95.00
\end{verbatim}

\begin{enumerate}
\def\labelenumi{\arabic{enumi})}
\setcounter{enumi}{3}
\tightlist
\item
  What is the distribution of the gender variable? Create a male dummy
  variable and indicate the distribution of this variable. Compare
  distribution of your male variable to the distribution of the gender
  variable.
\end{enumerate}

\begin{Shaded}
\begin{Highlighting}[]
\CommentTok{#Create male and female variables}
\NormalTok{lab}\OperatorTok{$}\NormalTok{male <-}\StringTok{ }\NormalTok{(lab}\OperatorTok{$}\NormalTok{gender }\OperatorTok{==}\StringTok{ }\DecValTok{1}\NormalTok{)}
\NormalTok{lab}\OperatorTok{$}\NormalTok{female <-}\StringTok{ }\NormalTok{(lab}\OperatorTok{$}\NormalTok{gender }\OperatorTok{==}\StringTok{ }\DecValTok{2}\NormalTok{)}
\end{Highlighting}
\end{Shaded}

Distribution of male variable

\begin{Shaded}
\begin{Highlighting}[]
\KeywordTok{table}\NormalTok{(lab}\OperatorTok{$}\NormalTok{male)}
\end{Highlighting}
\end{Shaded}

\begin{verbatim}
## 
## FALSE  TRUE 
##  1067   886
\end{verbatim}

Distribution of gender variable. The below table shows that 5
respondents did not identify with the binary gender definition

\begin{Shaded}
\begin{Highlighting}[]
\KeywordTok{table}\NormalTok{(lab}\OperatorTok{$}\NormalTok{gender)}
\end{Highlighting}
\end{Shaded}

\begin{verbatim}
## 
##    1    2    3 
##  886 1062    5
\end{verbatim}

\begin{enumerate}
\def\labelenumi{\arabic{enumi})}
\setcounter{enumi}{4}
\tightlist
\item
  Provide descriptive stats for Trump and Clinton feeling thermometer.
  Is there anything you need to adjust?
\end{enumerate}

\begin{Shaded}
\begin{Highlighting}[]
\CommentTok{#Summarize Clinton feeling thermometer, see there is a max of 200}
\KeywordTok{summary}\NormalTok{(lab}\OperatorTok{$}\NormalTok{therm_clinton)}
\end{Highlighting}
\end{Shaded}

\begin{verbatim}
##    Min. 1st Qu.  Median    Mean 3rd Qu.    Max.    NA's 
##    0.00   20.00   70.00   57.12   90.00  200.00     231
\end{verbatim}

\begin{Shaded}
\begin{Highlighting}[]
\CommentTok{#Turn values over 100 into NA and summarize again}
\NormalTok{lab}\OperatorTok{$}\NormalTok{therm_clinton[lab}\OperatorTok{$}\NormalTok{therm_clinton }\OperatorTok{>}\StringTok{ }\DecValTok{100}\NormalTok{] <-}\StringTok{ }\OtherTok{NA}
\KeywordTok{summary}\NormalTok{(lab}\OperatorTok{$}\NormalTok{therm_clinton)}
\end{Highlighting}
\end{Shaded}

\begin{verbatim}
##    Min. 1st Qu.  Median    Mean 3rd Qu.    Max.    NA's 
##    0.00   20.00   70.00   57.06   90.00  100.00     232
\end{verbatim}

\begin{Shaded}
\begin{Highlighting}[]
\CommentTok{#Summarize Trump feeling thermometer}
\KeywordTok{summary}\NormalTok{(lab}\OperatorTok{$}\NormalTok{therm_trump)}
\end{Highlighting}
\end{Shaded}

\begin{verbatim}
##    Min. 1st Qu.  Median    Mean 3rd Qu.    Max.    NA's 
##    0.00    0.00    0.00   17.76   25.00  100.00     292
\end{verbatim}

\begin{enumerate}
\def\labelenumi{\arabic{enumi})}
\setcounter{enumi}{5}
\tightlist
\item
  What is the distribution of the education variable? Is there any
  adjustment you would need to make if you will use this as a continuous
  variable in a regression model?
\end{enumerate}

\begin{Shaded}
\begin{Highlighting}[]
\CommentTok{#Below shows 7 values for education question}
\NormalTok{lab1 <-}\StringTok{ }\KeywordTok{read_dta}\NormalTok{(}\StringTok{"Ch2_lab_survey_data.dta"}\NormalTok{)}
\KeywordTok{table}\NormalTok{(lab1}\OperatorTok{$}\NormalTok{education)}
\end{Highlighting}
\end{Shaded}

\begin{verbatim}
## 
##   1   2   3   4   5   6   7 
##  17 125 245  11 134 677 746
\end{verbatim}

\begin{Shaded}
\begin{Highlighting}[]
\CommentTok{#Below we adjust education to exclude the Other response in answer 4, and re-assign the other responses accordingly}
\CommentTok{#Additionally, the Other response only had 11 values}
\NormalTok{lab1}\OperatorTok{$}\NormalTok{education[lab1}\OperatorTok{$}\NormalTok{education }\OperatorTok{==}\StringTok{ }\DecValTok{4}\NormalTok{] <-}\StringTok{ }\OtherTok{NA}
\NormalTok{lab1}\OperatorTok{$}\NormalTok{education[lab1}\OperatorTok{$}\NormalTok{education }\OperatorTok{==}\StringTok{ }\DecValTok{5}\NormalTok{] <-}\StringTok{ }\DecValTok{4}
\NormalTok{lab1}\OperatorTok{$}\NormalTok{education[lab1}\OperatorTok{$}\NormalTok{education }\OperatorTok{==}\StringTok{ }\DecValTok{6}\NormalTok{] <-}\StringTok{ }\DecValTok{5}
\NormalTok{lab1}\OperatorTok{$}\NormalTok{education[lab1}\OperatorTok{$}\NormalTok{education }\OperatorTok{==}\StringTok{ }\DecValTok{7}\NormalTok{] <-}\StringTok{ }\DecValTok{6}

\CommentTok{#Generate a new summary table}
\KeywordTok{table}\NormalTok{(lab1}\OperatorTok{$}\NormalTok{education)}
\end{Highlighting}
\end{Shaded}

\begin{verbatim}
## 
##   1   2   3   4   5   6 
##  17 125 245 134 677 746
\end{verbatim}

\begin{Shaded}
\begin{Highlighting}[]
\CommentTok{#A quick histogram to show the distribution of education}
\CommentTok{#Below (as well as the table above) shows a low number of responses from 1 - Some high school, which could indicate a bias in sampling, or simply reflect the education of the sample.}
\KeywordTok{hist}\NormalTok{(lab1}\OperatorTok{$}\NormalTok{education)}
\end{Highlighting}
\end{Shaded}

\includegraphics{Chapter2_Lab_sep17_files/figure-latex/unnamed-chunk-16-1.pdf}


\end{document}
